\documentclass[conference, 11pt]{IEEEtran}
 \IEEEoverridecommandlockouts
% The preceding line is only needed to identify funding in the first footnote. If that is unneeded, please comment it out.
\usepackage{cite}
\usepackage{amsmath,amssymb,amsfonts}
\usepackage{algorithmic}
\usepackage{graphicx}
\usepackage{textcomp}
\usepackage{xcolor}

\begin{document}

\title{Is comparing two completely different mobile devices nonsense?\\
% {\footnotesize \textsuperscript{*}Note: Sub-titles are not captured in Xplore and
% should not be used}
% \thanks{Identify applicable funding agency here. If none, delete this.}
}

\author{\IEEEauthorblockN{Omar Luis Duenas}
\IEEEauthorblockA{\textit{Institut für Telekooperation} \\
\textit{Johannes Kepler Universität Linz}\\
Linz, Österreich \\
roqueluisd@icloud.com}
\and
\IEEEauthorblockN{Lukas Wais}
\IEEEauthorblockA{\textit{Institut für Telekooperation} \\
\textit{Johannes Kepler Universität Linz}\\
Linz, Österreich \\
lukas.wais@outlook.de}
}

\maketitle

\begin{abstract}
In the past few years, a popular discussion that pops up every once in a while, revolves around smartphones or to be more specific: which smartphone is better? Nowadays, the market leaders in this field are Apple and Samsung. Every year they bring a new, revised version, of their leading products, where they improve almost everything. For the users it is inevitable to compare these products in each iteration. However, does comparing two different devices (iPhone 7 vs Samsung Galaxy S8) from two different companies make sense? In this paper, we will analyse the differences and similarities of these two popular smartphones. In order to reach an objective opinion on which of these smartphones is the best, we would conduct some relevant benchmarks in different areas.
\end{abstract}

\begin{IEEEkeywords}
mobile devices, performance, benchmarking, comparison
\end{IEEEkeywords}

\section{Introduction}
Nowadays nearly every month new flagship phones are being released. To find the most performant phone there are a lot of different performance tests. Are those tests meaningful? Or just random numbers that are generated by untrustworthy benchmark software. Moreover unlike desktop computers mobile device have much more versatile components. Moreover there are two main operating systems. Is it even possible to achieve a fair comparison? The last question we tried to tackle is, if the performance tests reflect day to day usage? We also defined the most importantly kind of performance tests for the end consumer to find the best smartphone for his or her needs.

\section{Similarities and Differences}
\subsection{Mobile Operating Systems in general}
An operation system in general is a system software which allowes the user to manage the hardware and software of your device. Therefore, if you talk about mobile operation systems (mobile OS), we are referring to a special OS design and program for mobile devices such as smartphones, tablets, smartwatches, etc. Laptops used to be part of the desktop OS implementation, but differences are becoming blurred in newer OS that are made for both uses (hybrid). 
The biggest difference between desktop OS and mobile OS is the I/O Management, than for mobile devices a mouse or keyboard is not needed. Moreover, for mobile OS some special features are implemented, and nowadays the most of them are considered essential. For example: wireless inbuilt modem, data connection and SIM try for telephony, touchscreen, Bluetooth, WI-FI, camera, speech recognition, Global Positioning System (GPS), etc.
As smartphones and mobile devices in general are getting more popular every day, the traditional desktop OS is a minority used kind of OS.
\subsection{Android}
The Android operating system was designed by the Open Handset Alliance
(led primarily by Google) and was developed for Android smartphones and
tablet computers. Whereas iOS is designed to run on Apple mobile devices
and is close-sourced, Android runs on a variety of mobile platforms and is
open-sourced, partly explaining its rapid rise in popularity. The structure of
Android appears in Figure 1  \cite{silberschatz}
\subsection{IOS}
iOS is a mobile operating system designed by Apple to run its smartphone, the iPhone, as well as its tablet computer, the iPad. iOS is structured on the Mac OS X operating system, with added functionality pertinent to mobile devices, but does not directly run Mac OS X applications. The structure of iOS appears in Figure 2.1. \cite{silberschatz}
\subsection{Comparison and Differences}
The architectures of the two operating systems are really different. iOS is based on a desktop operating system, which is based on BSD. The kernel of Android is Linux. The Linux kernel is not as focused on the end user computers like OSX from Apple is.

\begin{figure}[htbp]
\centerline{\includegraphics[width=0.5\textwidth]{android}}
\caption{architecture of Android\cite{silberschatz}}
\label{Figure 1}
\end{figure}

\begin{figure}[htbp]
\centerline{\includegraphics[width=0.5\textwidth]{osx}}
\caption{architecture of Mac OS \cite{silberschatz}}
\label{Figure 2}
\end{figure}

\begin{figure}[htbp]
\centerline{\includegraphics[width=0.20\textwidth]{ios}}
\caption{architecture of iOS \cite{silberschatz}}
\label{Figure 2.1}
\end{figure}

\section{Performance Tests}
\subsection{Why making performance tests}
The execution of performance tests have mostly economic reasons.
There are two different viewing points of testing and comparing mobile devices. The manufacture’s and the consumer’s view. The manufacturer’s aim is to be on the first in the ranking of the most popular performance tests. The fabricators want their flagship phones to be number one in the rankings to gain the maximum value of the media representations. 
On the other hand, the consumer’s view is to find and compare the best matching smartphone for his or her needs.

\subsection{Types of Performance Tests and how they work}
First of all you can compare and test nearly every aspect of a mobile device. \newline
The most important ones are: 
\begin{itemize}
\item Display $\rightarrow$ what the user sees.
\item GPU $\rightarrow$ the overall speed and graphical performance for gaming and performance demanding applications like image processing.
\item CPU $\rightarrow$  computing and overall speed, especially when starting new applications and multitasking.
\item Battery $\rightarrow$  how long the device lasts without charging.
\end{itemize}

\subsubsection{Display}
The easiest way to compare displays is by looking at the raw numbers, the screen resolution, the brigthness, the colorspace of the screen and so on.

\subsubsection{GPU}
A common method to test the graphical processing unit's performance is the calculation of The fast Fourier transform. \newline
The fast Fourier transform is a computational tool which facilitates signal analysis such as power spectrum analysis and filter simulation by means of digital computers. It is a method for efficiently computing the discrete Fourier transform of a series of data samples (referred to as a time series). \cite{fourier}

\subsubsection{CPU}
The performance testing of the central processing unit works similar to the GPU. A calculation of a heavy computational task, like the Dijkstra algorithm. \newline
The Dijkstra method is a well-known algorithm for finding the optimum path in shortest-path search problems. With that method, however, the time required to find the optimum path becomes remarkably long when the search scope is broad, so the Djikstra method is not suitable for real-time problems.
\cite{dij}

\subsubsection{Battery}
The main resource for battery testing is time. The mobile device repetitive performs day to day tasks like watching videos, or browsing the web. During this scenario the time is recording and the clock will stop when the phone shuts down due to low battery.

\subsection{Downside of comparisons}
The downside of the GPU performance tests is, that you have another layer added to the comparison, the graphical API of Android or iOS platforms. 
You do not compare directly the GPU, there is an extra layer between the raw hardware performance. The API from Apple is Metal and for Android it is Vulkan. It would not be such an unfair comparison, since these interfaces are software based not every device has them. Software should not interfere with the hardware performance.
\begin{figure}[htbp]
\centerline{\includegraphics[width=0.5\textwidth]{metal}}
\caption{the metal API from Apple \cite{metal}}
\label{Figure 3}
\end{figure}

With the CPU and also the GPU you have a problem with heat. Modern Smartphones are fan less, which means they are dealing with heat problems. Overall performance tests have to deal with various influences, like how many apps are running in the background, is the device charged or the temperature of the environment. It is not possible to reproduce the exact test result is shown on table 2. 
On the other hand these raw numbers do not depict real day to day usage. It is definitely not a common task to calculate The fast Fourier transform or running the Dijkstra algorithm.

\subsection{Geekbench 4}
There are various kinds of performance tests and different providers. Geekbench 4 is one of the most common tools to be used in performance testing. One of its key feature is that this software compares Android and iOS smartphones in the most comparable way possible. This means that the test results are calculated in the most abstract way possible. Two operating systems have a very different architecture, however Geekbench 4 tries to achieve a fair comparison.  
Geekbench 4 has three different kinds of benchmarks:
\begin{itemize}
\item CPU
\item Compute, this is a test for comparing the graphical power of the devices 
\item Battery
\end{itemize}
The CPU benchmarks works as follows: \newline
Geekbench 4 groups CPU workloads into two sections:
\begin{enumerate}
\item Single-Core Workloads
\item Multi-Core Workloads
\item Memory Workloads
\end{enumerate}

Each section is grouped into four subsections:
\begin{enumerate}
\item Cryptography Workloads
\item Integer Workloads
\item Floating-Point Workloads
\end{enumerate}
Geekbench inserts a pause (or gap) between each workload to minimize the effect thermal issues have on workload performance. Without this gap, workloads that appear later in the benchmark would have lower scores than workloads that appear earlier in the benchmark.The default gap is 2 seconds for both single-core and multi-core workloads. 
\begin{table}[htbp]
\centering
\begin{tabular}{|l|r|} 
\hline
Subsection     & Weight  \\ 
\hline
Cryptography   & 5\%     \\ 
\hline
Integer        & 45\%    \\ 
\hline
Floating Point & 30\%    \\ 
\hline
Memory         & 20\%    \\
\hline
\end{tabular}
\\
\vspace{5mm}
Table 1
\end{table}
\cite{geek}
\subsection{Results}
All tests were performed on the same Samsung Galaxy S8 with Android 9 with an Samsung Exynos 8895 CPU. The Version of Geekbench 4 is 4.3.2.
Every Application in the background was before the test closed.
\begin{table}[htbp]
\centering
\begin{tabular}{|l|c|c|c|}
\hline
CPU temperature     & 28 °C                     & 15 °C                     & 29 °C                     \\ \hline
Battery temperature & 26 °C                     & 13 °C                     & 26 °C                     \\ \hline
Memory usage        & 68 \%                     & 64 \%                     & 59 \%                     \\ \hline
charging            & no                        & no                        & yes                       \\ \hline
\multicolumn{4}{|c|}{Results}                                                                           \\ \hline
Multicore           & \multicolumn{1}{l|}{5713} & \multicolumn{1}{l|}{6467} & \multicolumn{1}{l|}{6125} \\ \hline
Singlecore          & \multicolumn{1}{l|}{1963} & \multicolumn{1}{l|}{1997} & \multicolumn{1}{l|}{2003} \\ \hline
\end{tabular}
\\
\vspace{5mm}
Table 2
\end{table}

Similar results can also be achieved with iPhones, in our test we used an iPhone 8 Plus. Due to the lack of user accessible CPU thermometers we can not find a correlation between the different results.
\subsection{Performance which really matters}
Most of the applications developed today are made with web technologies and are so called web apps. The main reason for this is to be platform independent. On the downside the web apps are not as performant as native developed apps, due to the extra abstraction layer for independencie. \newline \newline
Dedicated mobile web applications refer to web applications that are designed and developed to mimic the native applications of the host operating system as much as possible, but they execute in a web browser on the host platform. Dedicated mobile web applications are developed with a combination of HTML5, JavaScript, and CSS. \cite{webapp}
\newline
In the following pie chart you can see how many applications are made with this technique. 
In the ionic dev survey 2018 Developers were asked: 
\textit{Which of the following mobile development tools, libraries, and frameworks have you done extensive development work in over the past year?}\cite{ionic}
\begin{figure}[htbp]
\centerline{\includegraphics[width=0.5\textwidth]{survey}}
\caption{result of the ionic developer survey \cite{ionic}}
\label{Figure 5}
\end{figure}
\newpage
\section{Conclusion}
According to recent developments cross platform applications will increase on the app market. The performance in the everyday scenarios like browsing the web and scrolling through social media depends on the implementation of the hybrid app frameworks. It is very important not overload applications with unnecessary libraries consuming a lot of power in the background. For many apps it is not really essential for most use cases to have for example very fast 3D rendering. Nevertheless performance tests of mobile phones are giving the consumer a better overview and comparison of the smartphone market to identify the best performer.

%  \begin{thebibliography}{00}
%  \bibitem{b1} G. Eason, B. Noble, and I. N. Sneddon, ``On certain integrals of Lipschitz-Hankel type involving products of Bessel functions,'' Phil. Trans. Roy. Soc. London, vol. A247, pp. 529--551, 
%  April  1955.
% \end{thebibliography}

\bibliographystyle{IEEEtran}
\bibliography{references}


\end{document}
